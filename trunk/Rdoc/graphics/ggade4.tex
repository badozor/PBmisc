\documentclass[a4paper,10pt]{article}
\usepackage[utf8]{inputenc}
\usepackage{graphicx}
\usepackage{fancyvrb}
\usepackage{amsmath}
%\usepackage{mathtools}
\usepackage{color}
\usepackage{float}
\usepackage{hyperref}
\usepackage{tikz}


\title{\textbf{Working notes} \includegraphics[width=1cm]{/export/users/pierre/Images/Rlogo.png}\\ggplot2 implementation of the graphical functions of the ade4 package\\(in working!)}
\author{P.BADY}
\date\today




% paramter for tikz and organization and format of the diagrams
\usetikzlibrary{arrows,shapes,positioning,shadows,trees}
\tikzset{
  basic/.style  = {draw, text width=2cm, drop shadow, font=\sffamily, rectangle},
  root/.style   = {basic, rounded corners=2pt, thin, align=center,
                   fill=green!30},
  level 2/.style = {basic, rounded corners=6pt, thin,align=center, fill=green!60,
                   text width=8em},
  level 3/.style = {basic, thin, align=left, fill=pink!60, text width=6.5em}
}
% end

\usepackage{Sweave}
\begin{document}
\Sconcordance{concordance:ggade4.tex:ggade4.Rnw:%
1 17 1 1 9 13 1 1 0 54 1 1 24 26 0 1 2 2 1 1 2 1 0 1 4 6 0 1 2 5 1 1 2 1 0 %
2 1 3 0 1 2 3 1 1 2 1 0 4 1 4 0 1 2 8 1 1 2 1 0 8 1 1 2 1 0 2 1 4 0 1 2 10 %
1 1 2 1 0 13 1 4 0 1 2 14 1 1 2 1 0 9 1 4 0 1 2 12 1 1 2 1 0 10 1 4 0 1 2 7 %
1 1 2 1 0 10 1 4 0 1 2 11 1 1 2 1 0 11 1 4 0 1 2 6 1 1 2 1 0 11 1 4 0 1 2 7 %
1 1 16 18 0 1 2 3 1 1 2 1 0 13 1 1 4 3 0 2 1 5 0 1 3 10 1 1 8 7 0 1 9 11 0 %
1 2 4 1 1 2 1 0 1 2 1 0 2 1 1 3 2 0 5 1 4 0 1 2 7 1 1 2 1 0 1 2 1 0 2 1 1 3 %
2 0 5 1 4 0 1 2 10 1 1 2 1 0 10 1 4 0 1 2 8 1 1 2 1 0 10 1 4 0 1 2 8 1 1 2 %
1 0 10 1 4 0 1 2 11 1 1 2 1 0 9 1 4 0 1 2 9 1 1 2 1 0 2 1 1 2 1 0 1 1 1 2 1 %
0 5 1 4 0 1 2 6 1 1 2 1 0 1 1 3 0 1 2 3 1 1 2 1 0 1 1 1 2 1 0 6 1 4 0 1 2 %
34 1 1 2 59 0 1 1 4 0 1 3 1 1}

\definecolor{Soutput}{rgb}{0,0,0.56}
\definecolor{Sinput}{rgb}{0.56,0,0}
\DefineVerbatimEnvironment{Sinput}{Verbatim}
{formatcom={\color{Sinput}},fontsize=\footnotesize, baselinestretch=0.75}
\DefineVerbatimEnvironment{Soutput}{Verbatim}
{formatcom={\color{Soutput}},fontsize=\footnotesize, baselinestretch=0.75}
% Code de Duncan Murdoch post sur R-help le 07-MARS-2008 :
% This removes the extra spacing after code and output chunks in Sweave,
% but keeps the spacing around the whole block.
\DefineVerbatimEnvironment{GenericCode}{Verbatim}
{fontsize=\footnotesize, baselinestretch=0.75,frame=single}
\fvset{listparameters={\setlength{\topsep}{0pt}}}
\renewenvironment{Schunk}{\vspace{\topsep}}{\vspace{\topsep}}


\maketitle


\newlength\tindent
\setlength{\tindent}{\parindent}
\setlength{\parindent}{0pt}
\renewcommand{\indent}{\hspace*{\tindent}}


\rule{\linewidth}{.5pt}
\begin{center}
\begin{minipage}{0.9\textwidth}
% police par defaut verbatim
\fontfamily{\ttdefault}\selectfont
License: GPL version 2 or newer

Copyright (C) 2000-2021  Pierre Bady

This program/document is free software; you can redistribute it and/or modify it under the terms of the GNU General Public License as published by the Free Software Foundation; either version 2 of the License, or (at your option) any later version.

This program/document is distributed in the hope that it will be useful, but WITHOUT ANY WARRANTY; without even the implied warranty of MERCHANTABILITY or FITNESS FOR A PARTICULAR PURPOSE.  See the GNU General Public License for more details.
\end{minipage}\;
\end{center}
\rule{\linewidth}{.5pt}


\tableofcontents


\section{Motivations}

The objectives of this documment are to propose elements and alternatives for \texttt{ggplot2} implementation (\cite{ggplot2}) of the graphical function from R package ADE-4 (\cite{ade4_2004,ade4_2007a,ade4_2007b}). why ggplot2 ?


\section{Gestion of the limits}

A function \texttt{getLimits} is written to extract the limits of the axes x and y as done in the \texttt{R} package \texttt{ade4}.

\begin{Schunk}
\begin{Sinput}
  getLimits <- function (dfxy, xax=1, yax=2,include.origin=TRUE,origin=c(0,0)){
 	df <- data.frame(dfxy)
 	if (!is.data.frame(df)) 
 		stop("Non convenient selection for df")
 	if ((xax < 1) || (xax > ncol(df))) 
 		stop("Non convenient selection for xax")
 	if ((yax < 1) || (yax > ncol(df))) 
 		stop("Non convenient selection for yax")
 	x <- df[, xax]
 	y <- df[, yax]
 	x1 <- x
 	if (include.origin) 
 		x1 <- c(x1, origin[1])
 	x1 <- c(x1 - diff(range(x1)/10), x1 + diff(range(x1))/10)
 	xlim <- range(x1)
 	
 	y1 <- y
 	if (include.origin) 
 		y1 <- c(y1, origin[2])
 	y1 <- c(y1 - diff(range(y1)/10), y1 + diff(range(y1))/10)
 	ylim <- range(y1)
 	return(list(xlim=xlim, ylim=ylim))
 }
\end{Sinput}
\end{Schunk}

Example for a futur prototype of the function \texttt{ggade} (or \texttt{ggscatter}?)

\begin{Schunk}
\begin{Sinput}
  require(ggplot2)
  ggade <- function(dfxy,xax=1,yax=2,...,include.origin=TRUE,origin=c(0,0)){
   yxlim <- getLimits(dfxy,xax=xax,yax=yax,include.origin=include.origin,origin=origin)
   ggplot(....) + coord_cartesian(xlim=yxlim$xlim,ylim=yxlim$ylim) + coord_fixed(ratio=1)
 }
\end{Sinput}
\end{Schunk}


\section{Representations of the variables}



\begin{Schunk}
\begin{Sinput}
  data(deug)
  deug0 <- dudi.pca(deug$tab, center = deug$cent, scale = FALSE, scan = FALSE)
  deug1 <- dudi.pca(deug$tab, center = TRUE, scale = TRUE, scan = FALSE)
\end{Sinput}
\end{Schunk}


\begin{figure}[H]
\begin{center}
\begin{Schunk}
\begin{Sinput}
  require(ggplot2)
  gg <- ggplot(data=data.frame(eig=deug1$eig,nf=1:length(deug1$eig)), aes(x=nf, y=eig)) 
  gg <- gg + geom_bar(stat="identity") + ggtitle("Eigenvalues from PCA") 
  gg <- gg + ylab("Eigenvalues") + xlab("axis") + theme_light() 
  gg
\end{Sinput}
\end{Schunk}
\includegraphics{figs/sweave-eigen1}
\caption{Representation of the Eigenvalues from PCA on correlation matrix}
\label{fig:eigen1}
\end{center}
\end{figure}



\begin{figure}[H]
\begin{center}
\begin{Schunk}
\begin{Sinput}
  require(cowplot)
  require(ggplot2)
  require(ggrepel)
  auxi <- deug0$co
  auxi$label <- rownames(auxi)
  ggx <- ggplot(data=auxi,aes(Comp1,Comp2,label=label) )
  ggx <- ggx + geom_hline(yintercept = 0)+geom_vline(xintercept = 0)
  ggx <- ggx + xlab("axis 1") + ylab("axis 2")
  ggx <- ggx + geom_segment(aes(x=0,xend =Comp1, y=0,yend = Comp2),arrow = arrow(length = unit(0.2,"cm")))
  ggx <- ggx + geom_label_repel(size = 3.5,segment.alpha=0.7,segment.color = "darkgrey",
                               fontface = 'bold',col="black")
  ggx <- ggx + theme_bw() + coord_fixed(ratio=1)
  ggx
\end{Sinput}
\end{Schunk}
\includegraphics{figs/sweave-coarrow}
\caption{Representation of the variables with correlation circle for PCA on covariance}
\label{fig:coarrow}
\end{center}
\end{figure}





\begin{figure}[H]
\begin{center}
\begin{Schunk}
\begin{Sinput}
  require(cowplot)
  require(ggplot2)
  require(ggrepel)
  require(ggforce)
  auxi <- deug1$co
  auxi$label <- rownames(auxi)
  ggx <- ggplot(data=auxi,aes(Comp1,Comp2,label=label) )
  ggx <- ggx + geom_hline(yintercept = 0)+geom_vline(xintercept = 0)
  ggx <- ggx + xlab("axis 1") + ylab("axis 2")
  ggx <- ggx + geom_circle(data=data.frame(x0=0,y0=0),aes(x0=x0, y0=y0, r=1),inherit.aes = FALSE)
  ggx <- ggx + geom_segment(aes(x=0,xend =Comp1, y=0,yend = Comp2),arrow = arrow(length = unit(0.2,"cm")))
  ggx <- ggx + geom_label_repel(size = 3.5,segment.alpha=0.7,segment.color = "darkgrey",fontface = 'bold',col="black")
  ggx <- ggx + theme_bw() + coord_fixed(ratio=1)
  ggx
\end{Sinput}
\end{Schunk}
\includegraphics{figs/sweave-cocircle}
\caption{Representation of the variables with correlation circle for PCA on correlation}
\label{fig:cocircle}
\end{center}
\end{figure}





\section{Representations of samples}

representation of the samples on the first factorial plan

\begin{figure}[H]
\begin{center}
\begin{Schunk}
\begin{Sinput}
  require(ggplot2)
  require(ggrepel)
  auxi <- deug1$li
  auxi$label <- rownames(auxi)
  gg <- ggplot(auxi,aes(Axis1,Axis2,label=label))
  gg <- gg + geom_hline(yintercept = 0) + geom_vline(xintercept = 0)
  gg <- gg + geom_point(shape=19,size=3)
  gg <- gg + geom_label_repel(size = 3,segment.alpha=0.7,segment.color = "darkgrey")
  gg <-  gg + theme_bw()+ coord_fixed(ratio=1)
  gg
\end{Sinput}
\end{Schunk}
\includegraphics{figs/sweave-lilabel}
\caption{Representation of the observation of the first vectorial plan}
\label{fig:lilabel}
\end{center}
\end{figure}


\section{Representations of samples by group}
\subsection{separated Representations of samples by group}

Representation of the samples on the first factorial plan

\begin{figure}[H]
\begin{center}
\begin{Schunk}
\begin{Sinput}
  require(ggplot2)
  require(ggrepel)
  auxi <- deug1$li
  auxi$label <- rownames(auxi)
  auxi$group <- substring(as.character(deug$result),1,1)
  gg <- ggplot(auxi,aes(Axis1,Axis2,label=label))
  gg <- gg + geom_hline(yintercept = 0) + geom_vline(xintercept = 0)
  gg <- gg + geom_point(shape=19,size=3)
  gg <- gg + geom_label_repel(size = 3,segment.alpha=0.7,segment.color = "darkgrey")
  gg <-  gg + theme_bw() + coord_fixed(ratio=1) + facet_wrap(~ group) 
  gg
\end{Sinput}
\end{Schunk}
\includegraphics{figs/sweave-liclass1}
\caption{Representation of the observation of the first vectorial plan stratified by group.}
\label{fig:liclass1}
\end{center}
\end{figure}


\begin{figure}[H]
\begin{center}
\begin{Schunk}
\begin{Sinput}
  require(ggplot2)
  require(ggrepel)
  auxi <- deug1$li
  auxi$label <- rownames(auxi)
  auxi$group <- substring(as.character(deug$result),1,1)
  gg <- ggplot(auxi,aes(Axis1,Axis2,label=label,color=group))
  gg <- gg + geom_hline(yintercept = 0) + geom_vline(xintercept = 0)
  gg <- gg + geom_point(shape=19,size=3)
  gg <- gg + geom_label_repel(size = 3,segment.alpha=0.7,segment.color = "darkgrey")
  gg <-  gg + theme_bw() + coord_fixed(ratio=1) + facet_wrap(~ group)
  gg
\end{Sinput}
\end{Schunk}
\includegraphics{figs/sweave-liclass2}
\caption{Representation of the observation of the first vectorial plan stratified by group (with color).}
\label{fig:liclass2}
\end{center}
\end{figure}



\subsection{Ellipses}


\begin{figure}[H]
\begin{center}
\begin{Schunk}
\begin{Sinput}
  require(ggplot2)
  require(ggrepel)
  auxi <- deug1$li
  auxi$label <- rownames(auxi)
  auxi$group <- substring(as.character(deug$result),1,1)
  gg <- ggplot(auxi,aes(Axis1,Axis2,label=label,color=group))
  gg <- gg + geom_hline(yintercept = 0) + geom_vline(xintercept = 0)
  gg <- gg + geom_point(shape=19,size=3)
  gg <- gg + stat_ellipse(type = "norm",level=0.66)
  gg <- gg + geom_label_repel(size = 3,segment.alpha=0.7,segment.color = "darkgrey")
  gg <-  gg + theme_bw() + coord_fixed(ratio=1)
  gg
\end{Sinput}
\end{Schunk}
\includegraphics{figs/sweave-liclass3}
\caption{Representation of the observation of the first vectorial plan with ellpise of Inertia for each group}
\label{fig:liclass3}
\end{center}
\end{figure}

\begin{figure}[H]
\begin{center}
\begin{Schunk}
\begin{Sinput}
  require(ggplot2)
  require(ggrepel)
  auxi <- deug1$li
  auxi$label <- rownames(auxi)
  auxi$group <- substring(as.character(deug$result),1,1)
  gg <- ggplot(auxi,aes(Axis1,Axis2,label=label,color=group))
  gg <- gg + geom_hline(yintercept = 0) + geom_vline(xintercept = 0)
  gg <- gg + geom_point(shape=19,size=3)
  gg <- gg + stat_ellipse(type = "norm",level=0.66)
  gg <- gg + geom_label_repel(size = 3,segment.alpha=0.7,segment.color = "darkgrey")
  gg <-  gg + theme_bw() + coord_fixed(ratio=1) + facet_wrap(~group)
  gg
\end{Sinput}
\end{Schunk}
\includegraphics{figs/sweave-liclass4}
\caption{Representation of the observation of the first vectorial plan with ellpise of Inertia for each group}
\label{fig:liclass4}
\end{center}
\end{figure}

\subsection{Ellipses and stars}
For this representation, we need to compute the position of the gravity center of each ellipse. This operation was performed by the R function \texttt{prep.tab.class}. The implementation of this function is given below:

\begin{Schunk}
\begin{Sinput}
  prep.tab.class <- function(x,fac,varnames=c("Axis1","Axis2"),rm.X=TRUE,...){
   x <- as.data.frame(x)
   x$fac <- factor(x[,fac])
   w <- tab.class(x[,varnames],fac=x$fac)
   colnames(w) <- c("Axis1","Axis2")
   if(rm.X){
     w$label <- gsub("X","",rownames(w)) #levels(x$fac)
   }else{
     w$label <- rownames(w)
   }
   rownames(w) <- as.character(w$label)
   x$cAxis1 <- w$Axis1[match(as.character(x$fac),w$label)]
   x$cAxis2 <- w$Axis2[match(as.character(x$fac),w$label)]
   return(list(data=x,center=w))
 }
\end{Sinput}
\end{Schunk}


\begin{figure}[H]
\begin{center}
\begin{Schunk}
\begin{Sinput}
  require(ggplot2)
  require(ggrepel)
  auxi <- deug1$li
  auxi$label <- rownames(auxi)
  auxi$group <- substring(as.character(deug$result),1,1)
  w <- prep.tab.class(auxi,varnames=c("Axis1","Axis2"),fac="group",rm.X=FALSE)
  datax <- w$data
  centerx <- w$center
  gg <- ggplot(datax,aes(x=Axis1,y=Axis2,col=group,group=group)) 
  gg <- gg + geom_hline(yintercept = 0) + geom_vline(xintercept = 0)
  gg <- gg +  stat_ellipse(type = "norm",level=0.66,lwd=1,show.legend = FALSE)
  gg <- gg + geom_segment(data=datax,aes(x=Axis1,y=Axis2, xend = cAxis1, yend = cAxis2))
  gg <- gg + geom_point(data=centerx,aes(x=Axis1,y=Axis2,col=label),inherit.aes = FALSE,size=0)
  gg <- gg + geom_point(shape=19,size=2,show.legend = FALSE)
  # gg <- gg + geom_label_repel(data=centerx,aes(x=Axis1,y=Axis2,col=label,label=label),size = 3,
  #                             segment.alpha=0.7,segment.color = "darkgrey",fontface = 'bold',inherit.aes = FALSE)
  gg <- gg + geom_label(data=centerx,aes(x=Axis1,y=Axis2,col=label,label=label),size = 3,
                             fontface = 'bold',inherit.aes = FALSE)
  gg <- gg + theme_light() +theme(legend.position = "none") + coord_fixed(ratio=1)
  gg
  
\end{Sinput}
\end{Schunk}
\includegraphics{figs/sweave-liclassSTAR}
\caption{Representation of the observation of the first vectorial plan with ellpise of Inertia for each group}
\label{fig:liclassSTAR}
\end{center}
\end{figure}



\subsection{Chull representation}

The following code is based on the vignette related to the extending ggplot2 functions (https://cran.r-project.org/web/packages/ggplot2/vignettes/extending-ggplot2.htm). The additional functions for the computation necessary to the Convex hull representation are given below:

\begin{Schunk}
\begin{Sinput}
  StatChull <- ggproto("StatChull", Stat,
   compute_group = function(data, scales) {
     data[chull(data$x, data$y), , drop = FALSE]
   },
   
   required_aes = c("x", "y")
 )
  stat_chull <- function(mapping = NULL, data = NULL, geom = "polygon",
                        position = "identity", na.rm = FALSE, show.legend = NA, 
                        inherit.aes = TRUE, ...) {
   layer(
     stat = StatChull, data = data, mapping = mapping, geom = geom, 
     position = position, show.legend = show.legend, inherit.aes = inherit.aes,
     params = list(na.rm = na.rm, ...)
   )
 }
\end{Sinput}
\end{Schunk}

The representation for one group is given below:

\begin{figure}[H]
\begin{center}
\begin{Schunk}
\begin{Sinput}
  require(ggplot2)
  # data 
  auxi <- deug1$li
  auxi$label <- rownames(auxi)
  auxi$group <- substring(as.character(deug$result),1,1)
  # Find the convex hull of the points being plotted
  # Define the scatterplot
  gg <- ggplot(auxi,aes(Axis1,Axis2,col=group))
  gg <- gg + geom_hline(yintercept = 0) + geom_vline(xintercept = 0)
  gg <- gg + geom_point(shape=19,size=3)
  gg <- gg + stat_chull(fill=NA)
  gg <-  gg + theme_bw() + coord_fixed(ratio=1)
  gg
\end{Sinput}
\end{Schunk}
\includegraphics{figs/sweave-lichull1}
\caption{Chull representation of the observation of the first vectorial plan.}
\label{fig:lichull1}
\end{center}
\end{figure}


\begin{figure}[H]
\begin{center}
\begin{Schunk}
\begin{Sinput}
  require(ggplot2)
  # data 
  auxi <- deug1$li
  auxi$label <- rownames(auxi)
  auxi$group <- substring(as.character(deug$result),1,1)
  # Find the convex hull of the points being plotted
  # Define the scatterplot
  gg <- ggplot(auxi,aes(Axis1,Axis2,col=group,fill=group))
  gg <- gg + geom_hline(yintercept = 0) + geom_vline(xintercept = 0)
  gg <- gg + geom_point(shape=19,size=3)
  gg <- gg + stat_chull(alpha=0.5)
  gg <-  gg + theme_bw() + coord_fixed(ratio=1)
  gg
\end{Sinput}
\end{Schunk}
\includegraphics{figs/sweave-lichull2}
\caption{Chull representation of the observation of the first vectorial plan.}
\label{fig:lichull2}
\end{center}
\end{figure}



\section{2D density}

\begin{figure}[H]
\begin{center}
\begin{Schunk}
\begin{Sinput}
  require(ggplot2)
  require(ggrepel)
  auxi <- deug1$li
  auxi$label <- rownames(auxi)
  auxi$group <- deug$result
  gg <- ggplot(auxi,aes(Axis1,Axis2,label=label))
  gg <- gg + geom_hline(yintercept = 0) + geom_vline(xintercept = 0)
  gg <- gg + geom_point(shape=19,size=3)
  gg <- gg + stat_density2d(col="blue")
  gg <-  gg + theme_bw() + coord_fixed(ratio=1)
  gg
\end{Sinput}
\end{Schunk}
\includegraphics{figs/sweave-lidensity1}
\caption{Representation of the observation with 2-D kernel density}
\label{fig:lidensity1}
\end{center}
\end{figure}



\begin{figure}[H]
\begin{center}
\begin{Schunk}
\begin{Sinput}
  require(ggplot2)
  require(ggrepel)
  auxi <- deug1$li
  auxi$label <- rownames(auxi)
  auxi$group <- substring(as.character(deug$result),1,1)
  gg <- ggplot(auxi,aes(Axis1,Axis2,label=label,col=group))
  gg <- gg + geom_hline(yintercept = 0) + geom_vline(xintercept = 0)
  gg <- gg + geom_point(shape=19,size=3)
  gg <- gg + stat_density2d(col="blue")
  gg <-  gg + theme_bw() + coord_fixed(ratio=1)
  gg
\end{Sinput}
\end{Schunk}
\includegraphics{figs/sweave-lidensity2}
\caption{Representation of the observation with 2-D kernel density}
\label{fig:lidensity2}
\end{center}
\end{figure}



\begin{figure}[H]
\begin{center}
\begin{Schunk}
\begin{Sinput}
  require(ggplot2)
  require(ggrepel)
  auxi <- deug1$li
  auxi$label <- rownames(auxi)
  auxi$group <- substring(as.character(deug$result),1,1)
  gg <- ggplot(auxi,aes(Axis1,Axis2,label=label,col=group))
  gg <- gg + geom_hline(yintercept = 0) + geom_vline(xintercept = 0)
  gg <- gg + geom_point(shape=19,size=3)
  gg <- gg + stat_density2d(col="blue")
  gg <-  gg + theme_bw() + coord_fixed(ratio=1) + facet_wrap(~group)
  gg
\end{Sinput}
\end{Schunk}
\includegraphics{figs/sweave-lidensity3}
\caption{Representation of the observation with 2-D kernel density}
\label{fig:lidensity3}
\end{center}
\end{figure}



\section{Adding a picture}

The functions from the R package imager (\cite{imager}) can be used To add picture on ggplot graphics (\url{https://cran.r-project.org/web/packages/imager/vignettes/gettingstarted.html}).
\begin{figure}[H]
\begin{center}
\begin{Schunk}
\begin{Sinput}
  library(ggplot2)
  library(dplyr)
  library(imager)
  parrots <- load.image("/export/scratch/R/library/imager/extdata/parrots.png")
  df <- as.data.frame(parrots,wide="c") %>% mutate(rgb.val=rgb(c.1,c.2,c.3))
  df$y <- rev(df$y)
  gg <- ggplot(df,aes(x,y)) + geom_raster(aes(fill=rgb.val)) + scale_fill_identity()
  gg <- gg + xlab("x") + ylab("y")
  gg <- gg + theme_bw()  + coord_fixed(ratio=1)
  gg
\end{Sinput}
\end{Schunk}
\includegraphics{figs/sweave-image1}
\caption{Representation of "parrots" picture from R package \texttt{imager}.}
\label{fig:image1}
\end{center}
\end{figure}



Test for PNM format with imager ... resultats boff (?!)
\begin{figure}[H]
\begin{center}
\begin{Schunk}
\begin{Sinput}
  library(ggplot2)
  library(dplyr)
  library(imager)
  #photo1 <- load.image("/export/scratch/R/library/ade4/inst/pictures/buterfly.pnm")
  maps1 <- load.image("/export/scratch/R/library/ade4/pictures/butterfly.pnm")
  df <- as.data.frame(maps1)
  # black=#000000 and white = #FFFFFF
  df$rgb.val <- ifelse(df$value==0,'#000000','#FFFFFF')
  df$y <- rev(df$y)
  gg <- ggplot(df,aes(x,y)) + geom_raster(aes(fill=rgb.val)) + scale_fill_identity()
  gg <- gg + xlab("x") + ylab("y")
  gg <- gg + theme_bw()  + coord_fixed(ratio=1)
  gg
\end{Sinput}
\end{Schunk}
\includegraphics{figs/sweave-image2}
\caption{Representation of the map associated with butterfly data from R package \texttt{ade4}.}
\label{fig:image2}
\end{center}
\end{figure}

For the color code see https://www.nceas.ucsb.edu/sites/default/files/2020-04/colorPaletteCheatsheet.pdf. The previous results is not very good. we try to used the R package \texttt{magick} https://docs.ropensci.org/magick/articles/intro.html). The code is given below:


\begin{figure}[H]
\begin{center}
\begin{Schunk}
\begin{Sinput}
  require(magick)
  maps2 <-  image_read("/export/scratch/R/library/ade4/pictures/butterfly.pnm") 
  maps2
\end{Sinput}
\begin{Soutput}
# A tibble: 1 x 7
  format width height colorspace matte filesize density
  <chr>  <int>  <int> <chr>      <lgl>    <int> <chr>  
1 PGM      222    250 Gray       FALSE    55515 72x72  
\end{Soutput}
\end{Schunk}
\includegraphics{figs/sweave-image3}
\caption{Representation of the map associated with butterfly data from R package \texttt{ade4}.}
\label{fig:image3}
\end{center}
\end{figure}

add points and information
\begin{figure}[H]
\begin{center}
\begin{Schunk}
\begin{Sinput}
  data(butterfly)
  butterfly$xy
\end{Sinput}
\begin{Soutput}
      x   y
SS   41 238
SB   57 134
WSB  56 131
JRC  57 127
JRH  58 124
SJ   67 126
CR   73 121
UO  165  23
LO  167  18
DP   77 127
PZ   95  78
MC   72 186
IF  108 136
AF  106 128
GH  129 110
GL  113 141
\end{Soutput}
\begin{Sinput}
  gg <- ggplot(butterfly$xy,aes(x,y)) 
  gg <- gg + annotation_raster(as.raster(maps2),xmin=0,xmax=222,
                                         ymin=0,ymax=250,interpolate = TRUE)
  gg <- gg + geom_point()
  gg <- gg + xlab("x") + ylab("y")
  gg <- gg + theme_bw()  + coord_fixed(ratio=1) 
  gg <- gg + scale_y_continuous(limits=c(0,250)) 
  gg <- gg + scale_x_continuous(limits=c(0,222))
  gg
\end{Sinput}
\end{Schunk}
\includegraphics{figs/sweave-image4}
\caption{Representation of the map associated with butterfly data from R package \texttt{ade4}.}
\label{fig:image4}
\end{center}
\end{figure}


\section{Complex figures and Examples}

The complex figures are aggregated with the package \texttt{cowplot} (\cite{cowplot}).

\subsection{Heatmap and PCA}




\subsection{Coinertia analysis}




\subsection{K-table representations}






\section{Conclusion}


\bibliographystyle{plain}
\bibliography{ggade4bib}

\section{Appendix}

\begin{Schunk}
\begin{Sinput}
  print(sessionInfo(),locale=FALSE)
\end{Sinput}
\begin{Soutput}
R version 3.6.3 (2020-02-29)
Platform: x86_64-pc-linux-gnu (64-bit)
Running under: Linux Mint 18.3

Matrix products: default
BLAS:   /usr/lib/openblas-base/libblas.so.3
LAPACK: /usr/lib/libopenblasp-r0.2.18.so

Random number generation:
 RNG:     Mersenne-Twister 
 Normal:  Inversion 
 Sample:  Rounding 
 
attached base packages:
 [1] datasets  parallel  stats     graphics  utils     stats4    tools     grDevices
 [9] methods   base     

other attached packages:
 [1] magick_2.0           imager_0.41.2        magrittr_1.5        
 [4] dplyr_1.0.2          ggforce_0.2.2        ggrepel_0.8.1       
 [7] cowplot_0.9.4        ggplot2_3.3.2        knitr_1.23          
[10] pixmap_0.4-11        ade4_1.7-15          RColorBrewer_1.1-2  
[13] rtracklayer_1.44.0   GenomicRanges_1.36.0 GenomeInfoDb_1.20.0 
[16] IRanges_2.18.0       S4Vectors_0.22.1     BiocGenerics_0.30.0 

loaded via a namespace (and not attached):
 [1] Rcpp_1.0.5                  lattice_0.20-38            
 [3] png_0.1-7                   Rsamtools_2.0.0            
 [5] Biostrings_2.52.0           utf8_1.1.4                 
 [7] assertthat_0.2.1            digest_0.6.22              
 [9] R6_2.4.0                    tiff_0.1-5                 
[11] plyr_1.8.4                  pillar_1.4.7               
[13] zlibbioc_1.30.0             rlang_0.4.9                
[15] rstudioapi_0.10             Matrix_1.2-17              
[17] bmp_0.3                     labeling_0.3               
[19] BiocParallel_1.18.1         stringr_1.4.0              
[21] igraph_1.2.4.1              RCurl_1.95-4.12            
[23] polyclip_1.10-0             munsell_0.5.0              
[25] DelayedArray_0.10.0         compiler_3.6.3             
[27] xfun_0.10                   pkgconfig_2.0.3            
[29] readbitmap_0.1.5            tidyselect_1.1.0           
[31] SummarizedExperiment_1.14.0 tibble_3.0.4               
[33] GenomeInfoDbData_1.2.1      matrixStats_0.55.0         
[35] XML_3.98-1.20               fansi_0.4.0                
[37] crayon_1.3.4                withr_2.1.2                
[39] GenomicAlignments_1.20.0    MASS_7.3-51.4              
[41] bitops_1.0-6                grid_3.6.3                 
[43] gtable_0.3.0                lifecycle_0.2.0            
[45] scales_1.0.0                cli_1.1.0                  
[47] stringi_1.4.3               farver_1.1.0               
[49] XVector_0.24.0              ellipsis_0.3.0             
[51] generics_0.0.2              vctrs_0.3.5                
[53] Biobase_2.44.0              glue_1.4.2                 
[55] tweenr_1.0.1                purrr_0.3.4                
[57] jpeg_0.1-8.1                colorspace_1.4-1           
[59] isoband_0.2.3              
\end{Soutput}
\begin{Sinput}
  options(encoding="latin1",prompt=">  ", continue=" ", width = 85)
>  #save.image()
\end{Sinput}
\end{Schunk}

\end{document}
